% ==============================================================================
% DV017A
% Inledande Programmering i Java
% ---------------------------------
% Last updated 2015-06-16
%
% Author:
% Jonas Sjöberg     <tel12jsg@student.hig.se>
%
% License:
% Creative Commons Attribution-NonCommercial-ShareAlike 4.0 International
% See LICENSE.md for full licensing information.
% ==============================================================================

\documentclass[11pt,a4paper]{article}

\usepackage[utf8]{inputenc}
\inputencoding{utf8}
\usepackage[swedish]{babel}
\usepackage[T1]{fontenc}
\usepackage{lmodern}
\usepackage{fullpage}

\usepackage{textcomp}
\usepackage{url}
\usepackage{graphicx}

\usepackage{minted}
\usemintedstyle{pastie} % Bland Pygments themes är "pastie" favvo
                        % färgglad och "trac" favvo sober ..

\renewcommand\listingscaption{Program source code}
\renewcommand\listoflistingscaption{List of program sources}

\usepackage{verbatim}
\usepackage{listings}


\title{DV017A \\ Java-programmering \\ Laboration 1}

\author{\\
  Jonas Sjöberg\\
  Högskolan i Gävle,\\
  Elektronikingenjörsprogrammet,\\
  \texttt{tel12jsg@tudent.hig.se}
}

\date{}

\begin{document}
    \maketitle

    \begin{center}
    \begin{tabular}{l r}
        Datum: & Juni 2015 \\
        Kursansvarig lärare: & Atique Ullah
    \end{tabular}
    \end{center}

    \begin{abstract}
        Laboration i DV017A - Inledande programmering i Java. Blandade uppgifter
        med instruktioner, lösningar, källkod, skärmdumpar och kommentarer.
    \end{abstract}

    \newpage
    %\hypersetup{linkcolor=black}
    \setcounter{tocdepth}{3}
    \tableofcontents
    \newpage

%   \section{Introduction}\label{setup}
This is a introduction.

    \section{Uppgift 1}\label{uppgift-1}

\subsection{Instruktioner}
\begin{verbatim}
1. I nedanstående program saknas datatyperna (där det står ...) vid
   variabeldeklarationerna. Din uppgift är att fylla i rätt datatyp vid
   respektive deklaration. Provkör sedan programmet:

   public class Datatyper
   {
       public static void main(String[] args)
       {
           ... data1 = true;
           ... data2 = 45.8F;
           ... data3 = 29;
           ... data4 = data3 < 10;
           ... data5 = 12 / 5;
           ... data6 = data3 * data5;
           ... data7 = 10 % 3;
           ... data8 = "Java programmering";
           ... data9 = 'b';
           ... data10 = (float)data5 / 4;

             System.out.println     ("Variabeln     data1: " + data1);
             System.out.println     ("Variabeln     data2: " + data2);
             System.out.println     ("Variabeln     data3: " + data3);
             System.out.println     ("Variabeln     data4: " + data4);
             System.out.println     ("Variabeln     data5: " + data5);
             System.out.println     ("Variabeln     data6: " + data6);
             System.out.println     ("Variabeln     data7: " + data7);
             System.out.println     ("Variabeln     data8: " + data8);
             System.out.println     ("Variabeln     data9: " + data9);
             System.out.println     ("Variabeln     data10: " + data10);
         }
     }
\end{verbatim}

\subsection{Lösning}
\subsubsection{Kommentar}
Det är något oklart vad som efterfrågas. I problembeskrivningen specifieras
inte huruvida målvariabelns datatyp ska vara lämpad för att lagra resultatet av
beräkningen på det vis att ingen data går förlorad genom trunkering eller
avrundning.  Eller om målvariabeln helt enkelt ska ha samma datatyp som de
övriga operanderna. Jag har valt att vara konservativ i fråga om "resurser" 
och använder de datatyper som krävs för att lagra den information som ska
behandlas. Om exekversingstid och/eller lagringsutrymme inte är någon faktor att
beakta kunde alla variabler få vara 64-bitars floats.

%Bla \mintinline{latex}{\mintinline{latex}{your $code$ goes here}} bla.

\par Nämnvärt är att en inmatad siffra utan specifierad datatyp ges typen
integer
%per default. Detta är fallet för \mintinline{java}{data5 = 15 / 5;} : 12 och 5
%har typen int och
per default. Detta är fallet för $data5 = 15 / 5$ ; 12 och 5 har typen int och
resultatet innehåller inga decimaler.  Detaljer rörande 'literals' hämtas bäst
från The Java Language Specification, Java SE 8 Edition
\footnote{\url{https://docs.oracle.com/javase/specs/}}



\subsubsection{Källkod}\label{uppgift-1_src}
%\begin{listing}[]
    \inputminted[linenos]{java}{src/Lab1Uppg01.java}
    \caption{Lab1Uppg01.java}
    \label{Uppg1src}
%\end{listing}


    \section{Uppgift 2}\label{uppgift-2}

\subsection{Instruktioner}
\begin{verbatim}
2. Skriv ett program som skriver ut summan, medelvärdet och produkten av tre
   heltal.  De tre heltalen ska användaren skriva in från tangentbordet när
   programmet körs. Programmets utskrift kan t.ex se ut så här, det som skrivs
   in från tangentbordet är markerat med fetstil/understrykning:

    Skriv in tre heltal.
    Skriv in det första talet: *20*
    Skriv in det andra talet: *30*
    Skriv in det tredje talet: *25*
    Summan av talen är 75.
    Medelvärdet av talen är 25.
    Produkten av talen är 15000.
\end{verbatim}

\subsection{Lösning}
\subsubsection{Funktion}
Programmet använder \texttt{static final} i början av programmet för variabler
som inte ska ändras under exekvering.  Variabler med enbart versaler är en
slags globala variabler som kan modifieras av programmeraren.
\par Räkneord sparas i sträng-arrays \texttt{GRUNDTAL} och
\texttt{ORDNINGSTAL}, för att hämtas och skrivas ut vid exekvering.

\subsubsection{Kommentar}
\par Rad 33, \texttt{@SuppressWarnings("resource")} syftar till att undertrycka
varningar från Eclipse och gör ingen faktisk skillnad på programmets funktion.
\par Programmet saknar filtrering av indata, vilket är ett stort problem då det
innebär stora säkerhetsrisker och möjliga odefinerade, alternativt outforskade
skeenden. Mer om detta senare..

\subsubsection{Källkod}\label{uppgift-2_src}
%\begin{listing}[H]
    \inputminted[linenos]{java}{src/Lab1Uppg02.java}
    \caption{Lab1Uppg02.java}
    \label{Uppg2src}
%\end{listing}

    \section{Uppgift 3}\label{uppgift-3}

\subsection{Beskrivning}
TODO: Beskriving uppgift 3.

\subsection{Källkod}\label{uppgift-3_src}
%\subsubsection{Lab1Uppg03.java}
\inputminted[]{java}{../src/Lab1Uppg03.java}

    \section{Uppgift 4}\label{uppgift-4}

\subsection{Beskrivning}
TODO: Beskriving uppgift 4.

\subsection{Källkod}\label{uppgift-4_src}
%\subsubsection{Lab1Uppg04.java}
\inputminted[]{java}{../src/Lab1Uppg04.java}

    \section{Uppgift 5}\label{uppgift-5}

\subsection{Beskrivning}
TODO: Beskriving uppgift 5.

\subsection{Källkod}\label{uppgift-5_src}
%\subsubsection{Lab1Uppg05.java}
\inputminted[]{java}{../src/Lab1Uppg05.java}

    \section{Uppgift 6}\label{uppgift-6}

\subsection{Beskrivning}
TODO: Beskriving uppgift 6.

\subsection{Källkod}\label{uppgift-6_src}
%\subsubsection{Lab1Uppg06.java}
\inputminted[]{java}{../src/Lab1Uppg06.java}

    \section{Uppgift 7}\label{uppgift-7}

\subsection{Beskrivning}
\subsubsection*{Instruktioner}
\begin{verbatim}
7. Skriv ett program som låter användaren mata in värden till de tre
   heltalsvariablerna var1, var2 och var3. Programmet skall sedan för varje
   påstående a) – e) lagra värdet av påståendet i den booleska variabeln svar
   och därefter skriva ut värdet av svar , försett med lämplig ledtext:

    a) Talet var1 är jämnt delbart med 7.
    b) Talet var3 är inte jämnt delbart med talet var2.
    c) Talet var1 är större än minst något av talen var2 och var3.
    d) Talet var1 är större än talet var2, som i sin tur är större än talet var3.
    e) Talet var1 är större än ett av talen var2 och var3, men inte större än båda.

   Tips: För att kolla om något tal är jämnt delbart med ett annat,
         använd modulus-operatorn %!
\end{verbatim}

\subsection{Kommentar}
% TODO: Kommentar på %7.

\subsection{Källkod}\label{uppgift-7_src}
\subsubsection*{Lab1Uppg07.java}
\inputminted[]{java}{src/Lab1Uppg07.java}

    \section{Uppgift 8}\label{uppgift-8}

\subsection{Beskrivning}
TODO: Beskriving uppgift 8.

\subsection{Källkod}\label{uppgift-8_src}
%\subsubsection{Lab1Uppg08.java}
\inputminted[]{java}{../src/Lab1Uppg08.java}

    \section{Uppgift 9}\label{uppgift-9}

\subsection{Beskrivning}

\subsubsection*{Instruktioner}
\begin{verbatim}
9. Skriv ett program som summerar alla jämna tal från och med 0 t.o.m 200 med
   hjälp av en while-sats och sedan skriver ut denna summa. Skriv sedan två
   program till som utför samma sak, men som istället använder sig av en do-
   resp en for-sats. Observera att du ska skriva de tre programmen utan att
   använda någon if-sats inne i iterations-satserna.
\end{verbatim}

\subsection{Kommentar}
% TODO: Kommentar på #9.

\subsection{Källkod}\label{uppgift-9_src}
\subsubsection*{Lab1Uppg09.java}
\inputminted[]{java}{src/Lab1Uppg09.java}

    \section{Uppgift 10}\label{uppgift-10}

\subsection{Beskrivning}

\subsubsection*{Instruktioner}
\begin{verbatim}
10. Skriv ett program där användaren ska ange ordningsnumret på en månad (1-12).
    Programmet ska sedan skriva ut månadens namn och antal dagar. Om man anger
    felaktigt månadsnummer (mindre än 0 eller större än 12) så ska ett
    felmeddelande skrivas på skärmen, och man ska sedan kunna ange ett nytt
    nummer. Använd switch-sats i din lösning

    Programmet ska fortsätta att fråga efter månadsnummer ända tills man har
    matat in 0 (noll). Då ska programmet avbrytas.
\end{verbatim}


\subsection{Kommentar}
% TODO: Kommentar på #10.

\subsection{Källkod}\label{uppgift-10_src}
\subsubsection*{Lab1Uppg10.java}
\begin{listing}[H]
    \inputminted[]{java}{src/Lab1Uppg10.java}
    \caption{Lab1Uppg10.java}
    \label{Uppg10src}
\end{listing}

%   \section{Resultat}\label{setup}
TODO: Eventuellt resultat och kommentar på hela labben.


    \newpage

    \section{Referenser}\label{referenser}

\subsection{www}\label{}
% ==============================================================================

\subsection{Literature}\label{}
% ==============================================================================

\subsection{Source files}\label{sources}
% ==============================================================================
Full source, including spice simulation files, CSV data, schematics, etc
TODO: is available at \url{https://github.com/jonasjberg/FIX_URL}



\end{document}
