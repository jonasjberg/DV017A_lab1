\section{Uppgift 6}\label{uppgift-6}

\subsection{Beskrivning}
\subsubsection*{Instruktioner}
\begin{verbatim}
6. Skriv två program som ger samma utskrift som den i uppgift 5, men använd
   istället en for- loop i ena programmet och en do-loop i det andra.
\end{verbatim}

\subsection{Kommentar}
Programmet använder återigen samma logik för att kontrollera korrekt inmatning
men till skillnad från i Uppgift \ref{uppgift-4} och Uppgift \ref{uppgift-5}
används en funktion \texttt{getUserInput()}.
\par Koden blir modulär och enklare att hantera. Funktionens åtkomstmodifierare
\texttt{static} gör funktionen till en klassfunktion, alla objekt som instanstieras
från klassen delar funktionen. Funktionen hör till klassen, inte objekten.
\texttt{protected} gör functionen ''synlig'' inom paketet.
\par Själva nedräkningen görs också med funktioner. 
\texttt{protected static void countDownUsingForLoop(int start)} räknar ner från
parametervärdet \texttt{start} med hjälp av en \texttt{for}-loop.
\texttt{protected static void countDownUsingDoLoop(int start)} räknar ner från
parameter värdet \texttt{start} med hjälp av en \texttt{do}-loop.


\subsection{Källkod}\label{uppgift-6_src}
\subsubsection*{Lab1Uppg06.java}
%\begin{listing}[H]
    \inputminted[linenos]{java}{src/Lab1Uppg06.java}
    \caption{Lab1Uppg06.java}
    \label{Uppg6src}
%\end{listing}
